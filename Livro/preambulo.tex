%Este trabalho está licenciado sob a Licença Creative Commons Atribuição-CompartilhaIgual 3.0 Não Adaptada. Para ver uma cópia desta licença, visite https://creativecommons.org/licenses/by-sa/3.0/ ou envie uma carta para Creative Commons, PO Box 1866, Mountain View, CA 94042, USA.

%%%%%%%%%%%%%%%%%%%%%%%%%%%%%%%%%%%%%%%%%
% ATENÇÃO
%
% POR SEGURANÇA, NÃO EDITE ESTE ARQUIVO.
%
%%%%%%%%%%%%%%%%%%%%%%%%%%%%%%%%%%%%%%%%

%%%%%%%%%%%%%%%%%%%%%%%%%%%%%%%%%
%   Predefinicoes
%%%%%%%%%%%%%%%%%%%%%%%%%%%%%%%%%

\newif\ifisbook        % O layout será book?
\newif\ifishtml        % O layout será html?
\newif\ifisslide       % O layout será slide?

\newif\ifisscilab      % As notas incluirão scilab?
\newif\ifisoctave      % As notas incluirão octave?
\newif\ifispython      % As notas incluirão python?

\def\tfn{config.knd}     % Arquivo que guarda as definições do tipo de saída
\def \tdata{}          % Definições do tipo de saída: book, slide ou html.

\openin1=\tfn\relax    % Leitura das definições de saída
\read1 to \tdata
\closein1

\tdata                 % Definições de saída

%%%%%%%%%%%%%%%%%%%%%%%%%%%%%%%%%
%   Opcões de Linguagem
%%%%%%%%%%%%%%%%%%%%%%%%%%%%%%%%%
\usepackage[brazil]{babel}
\usepackage[utf8]{inputenc}
\usepackage[T1]{fontenc}
%\usepackage{xunicode} é o pacote necessário para a codificação UTF-8 no XeTeX

\usepackage{fancyhdr}
\pagestyle{fancy}
\fancyhf{}
\fancyhead[RE]{Cálculo Numérico}
\fancyhead[LO]{\rightmark}
\fancyhead[LE,RO]{\thepage}

%license footnote
\cfoot{Prof. M.e Daniel Cassimiro}
%\cfoot{\tiny{Licença \href{https://creativecommons.org/licenses/by-sa/3.0/}{CC-BY-SA-3.0}. Contato: \url{reamat@ufrgs.br}}}

%%%% no blank pages between chapters %%%%
\let\cleardoublepage\clearpage

%%%% independent chapters %%%%
\usepackage{subfiles}

%%%% ams-latex %%%%
\usepackage{amsmath}
\usepackage{amssymb}
\usepackage{amsthm}
\usepackage{mathtools}

%%%% graphics %%%%
\usepackage{graphics}
\usepackage{graphicx}
%\usepackage{caption}

%%%% links %%%%
\usepackage[pdfborder={0 0 0 [0 0]},colorlinks=true,linkcolor=blue,citecolor=blue,filecolor=blue,urlcolor=blue]{hyperref}

%%%% copy and paste from PDF (correctly) %%%%
\usepackage{upquote}
\usepackage{lmodern}

\usepackage[usenames]{color}%%%%%%%%%%%%%%%%%%%%%%%%%%%%%%%%%%%%%%%%%%%

%%%% code insert (verbatim) %%%%
\usepackage{verbatim}
\usepackage{listings}
\definecolor{dkgreen}{rgb}{0,0.6,0}
\definecolor{gray}{rgb}{0.5,0.5,0.5}
\definecolor{mauve}{rgb}{0.58,0,0.82}
\lstset{         
  basicstyle=\small\ttfamily,
  %basicstyle=\footnotesize,   
  language=Python,         
  numbers=left,                   
  numberstyle=\tiny\color{gray},  
  stepnumber=1,                             
  numbersep=5pt,                  
  %backgroundcolor=\color{white},    
  showspaces=false,               
  showstringspaces=false,         
  showtabs=false,                 
  frame=single,                   
  rulecolor=\color{black},        
  tabsize=2,                      
  captionpos=b,                   
  breaklines=true,                
  breakatwhitespace=false,        
  title=\lstname,                               
  keywordstyle=\color{blue},          
  commentstyle=\color{red},       
  stringstyle=\color{mauve},    
  morecomment=[s][\color{blue}]{/**}{*/},
  columns=fullflexible,
  otherkeywords = {>},
  morekeywords=[2]{julia, >},
  keywordstyle=[2]{\color{green!50!black}},
  xleftmargin=10pt,
  belowskip=-1.0 \baselineskip,
  texcl=true 
}





%%%% indent first line %%%%
\usepackage{indentfirst}

%%%% comma as a decimal separator %%%%
\usepackage{icomma}

%%%% citation %%%%
\usepackage{cite}

%%%% lists %%%%
\usepackage{enumerate}

%%%% miscellaneous %%%%
\usepackage{multicol}
\usepackage{multirow}
\usepackage[normalem]{ulem}
\renewcommand{\arraystretch}{1.5} %space between rows in tables
\usepackage{array,booktabs}

%%%%%%%%%%%%%%%%%%%%%%%%%%%%%%%%%
%   Formatacoes de estilo
%%%%%%%%%%%%%%%%%%%%%%%%%%%%%%%%%
\usepackage{xcolor}
\newcommand{\RED}[1]{{\color{red}{#1}}}
\newcommand{\BLU}[1]{{\color{blue}{#1}}}
\newcommand{\GRE}[1]{{\color{darkgreen}{#1}}}

\usepackage{tikz}
\newcommand*\circled[1]{\tikz[baseline=(char.base)]{
            \node[shape=circle,draw,inner sep=1pt] (char) {#1};}}

%emphasis \emph
\DeclareTextFontCommand{\emph}{\bfseries}

\newcommand{\sen}{\operatorname{sen}\,}
\newcommand{\senh}{\operatorname{senh}\,}
\renewcommand{\sin}{\operatorname{sen}\,}
\renewcommand{\sinh}{\operatorname{senh}\,}
\newcommand{\tg}{\operatorname{tg}\,}

%%%% indexing %%%%
\usepackage{makeidx}
\makeindex

%%%%% macros  %%%%%%%%%%%%%
\newcommand{\matdd}[4]{\begin{bmatrix} #1&#2\\#3&#4 \end{bmatrix}}
\newcommand{\matddd}[9]{\begin{bmatrix} #1&#2&#3 \\ #4&#5&#6 \\ #7&#8&#9 \end{bmatrix}}
\newcommand{\vetdd}[2]{\begin{bmatrix} #1 \\#2 \end{bmatrix}}
\newcommand{\vetddd}[3]{\begin{bmatrix} #1 \\ #2\\ #3 \end{bmatrix}}
\newcommand{\field}[1]{\mathbb{#1}}

\newcommand{\emconstrucao}{
  \begin{tabular}{|c|}\hline
    Em construção ... Gostaria de participar na escrita deste livro? Veja como em:\\
    \url{https://www.ufrgs.br/reamat/participe.html}\\\hline
  \end{tabular}
}

\newcommand{\construirSec}{
\begin{center}
  Esta seção (ou subseção) está sugerida. Participe da sua escrita. Veja como em:\\
  \url{https://github.com/livroscolaborativos/CalculoNumerico}
\end{center}
}

\newcommand{\construirExeresol}{
  \begin{center}
    \begin{tabular}{|c|}\hline
      Esta seção carece de exercícios resolvidos. Participe da sua escrita.\\
      Veja como em:\\
      \url{https://github.com/livroscolaborativos/CalculoNumerico}\\\hline
    \end{tabular}
  \end{center}
}

\newcommand{\construirExer}{
  \begin{center}
    \begin{tabular}{|c|}\hline
      Esta seção carece de exercícios. Participe da sua escrita.\\
      Veja como em:\\
      \url{https://github.com/livroscolaborativos/CalculoNumerico}\\\hline
    \end{tabular}
  \end{center}
}

\newcommand{\construirResp}{
\begin{center}
  Este exercício está sem resposta sugerida. Proponha uma resposta. Veja como em:\\
  \url{https://github.com/livroscolaborativos/CalculoNumerico}
\end{center}
}

\newcommand{\construirScilab}{
  \begin{center}
    \begin{tabular}{|c|}\hline
      Aqui, cabe um código Scilab explicativo. Escreva você mesmo o código.\\
      Veja como participar da escrita do livro em:\\
      \url{https://github.com/livroscolaborativos/CalculoNumerico}\\\hline
    \end{tabular}
  \end{center}
}
\newcommand{\construirOctave}{
  \begin{center}
    \begin{tabular}{|c|}\hline
      Aqui, cabe um código GNU Octave explicativo. Escreva você mesmo o código.\\
      Veja como participar da escrita do livro em:\\
      \url{https://github.com/livroscolaborativos/CalculoNumerico}\\\hline
    \end{tabular}
  \end{center}
}
\newcommand{\construirPython}{
  \begin{center}
    \begin{tabular}{|c|}\hline
      Aqui, cabe um código Python explicativo. Escreva você mesmo o código.\\
      Veja como participar da escrita do livro em:\\
      \url{https://github.com/livroscolaborativos/CalculoNumerico}\\\hline
    \end{tabular}
  \end{center}
}

\usepackage{amsmath,amssymb} %%%%%%%%%%%%%%%%%%%%%%%%%%%%%%%%%%%%%%%%%%
\usepackage{amsthm} %%%%%%%%%%%%%%%%%%%%%%%%%%%%%%%%%%%%%%%%%%%%%%%%%%%
\usepackage{enumerate} %%%%%%%%%%%%%%%%%%%%%%%%%%%%%%%%%%%%%%%%%%%%%%%%
\usepackage{longtable}%%%%%%%%%%%%%%%%%%%%%%%%%%%%%%%%%%%%%%%%%%%%%%%%%
\usepackage{hhline}%%%%%%%%%%%%%%%%%%%%%%%%%%%%%%%%%%%%%%%%%%%%%%%%%%%%
\usepackage{graphicx}%%%%%%%%%%%%%%%%%%%%%%%%%%%%%%%%%%%%%%%%%%%%%%%%%%
\usepackage{pict2e}%%%%%%%%%%%%%%%%%%%%%%%%%%%%%%%%%%%%%%%%%%%%%%%%%%%%
\usepackage{multicol}%%%%%%%%%%%%%%%%%%%%%%%%%%%%%%%%%%%%%%%%%%%%%%%%%%
%\usepackage{indentfirst} % Identação da primeira linha do capítulo %%%
\usepackage{hyperref} %% pacote hyperef cria link (e marcador no PDF) %
% regra de hifeniza\c{c}\~ao das palavras nao acentuadas %%%%%%%%%%%%%%
\hyphenation{li-vro tes-te cha-ve bi-blio-te-ca con-si-de-ran-do-se ca-sa} %%%%%%%%%%%%%%%%%%%%%%
% regra de hifeniza\c{c}\~ao para palavras acentuadas: %%%%%%%%%%%%%%%%
% requer \usepackage[T1]{fontenc} %%%%%%%%%%%%%%%%%%%%%%%%%%%%%%%%%%%%%
%\hyphenation{co-men-t\'a-rio re-fe-r\^en-cia} %%%%%%%%%%%%%%%%%%%%%%%%
%%%%%%%%%%%%%%%%%%%%%%%%%%%%%%%%%%%%%%%%%%%%%%%%%%%%%%%%%%%%%%%%%%%%%%%
\newcommand{\Cset}{\ensuremath{\mathbb{C}}}%%%%%%%%%%%%%%%%%%%%%%%%%%%%
\newcommand{\Rset}{\ensuremath{\mathbb{R}}}%%%%%%%%%%%%%%%%%%%%%%%%%%%%
\newcommand{\Qset}{\ensuremath{\mathbb{Q}}}%%%%%%%%%%%%%%%%%%%%%%%%%%%%
\newcommand{\Iset}{\ensuremath{\mathbb{I}}}%%%%%%%%%%%%%%%%%%%%%%%%%%%%
\newcommand{\Zset}{\ensuremath{\mathbb{Z}}}%%%%%%%%%%%%%%%%%%%%%%%%%%%%
\newcommand{\Nset}{\ensuremath{\mathbb{N}}}%%%%%%%%%%%%%%%%%%%%%%%%%%%%
\newtheorem{teorema}{Teorema}[subsection]
\newtheorem{obs}[teorema]{Observa\c{c}\~{a}o}
\newtheorem{defini}[teorema]{Defini\c{c}\~{a}o}
\newtheorem{prop}[teorema]{Proposi\c{c}\~{a}o}
\newtheorem{corolario}[teorema]{Corol\'{a}rio}
\newtheorem{lema}[teorema]{Lema}
\newtheorem{exercicio}[teorema]{Exerc\'{i}cio}
\newenvironment{demonstracao}{\noindent{\bf Demonstra\c{c}\~ao:} }{\hfill $\Box$ \newline}
\newenvironment{demontracaosembox}{\noindent{\bf Demonstra\c{c}\~ao:} }{}
\newenvironment{exemplos}{\noindent {\bf Exemplos:} }{\hfill $\Box$ \newline}
\newenvironment{exemplo}{\vspace{12pt} \noindent{\bf Exemplo:} }{\hfill $\Box$ \newline}
\newenvironment{exemplosembox}{\vspace{12pt} \noindent {\bf Exemplos:} }{}
\newenvironment{demteo}{\noindent {\bf Demonstra\c{c}\~{a}o do Teorema}}{\hfill $\Box$ \newline}
\newenvironment{demlem}{\noindent {\bf Demonstra\c{c}\~{a}o do Lema}}{\hfill $\Box$ \newline}

%\usepackage{afterpage} % executar comando após finalizar a página
\usepackage{multicol} % permite usar multiplas linhas
% espessura da linha que separa as colunas (0pt para desabilitar)
\setlength{\columnseprule}{1pt} 
% cor da linha que separa as colunas
\def\columnseprulecolor{\color{white}}
%\pagestyle{headings}
\usepackage{lipsum}
\usepackage{import}
\usepackage{pgfplots}
%\pgfplotsset{compat=1.15}
\usepackage{mathrsfs}
\usetikzlibrary{arrows}


%E = 10^
\def\E#1{\mathrm{E}\!#1\!}

\ifisslide
\input preambulo_slide.tex
\else
\ifishtml
\input preambulo_html.tex
\else
\input preambulo_book.tex
\fi
\fi