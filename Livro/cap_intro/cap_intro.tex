%Este trabalho está licenciado sob a Licença Creative Commons Atribuição-CompartilhaIgual 3.0 Não Adaptada. Para ver uma cópia desta licença, visite https://creativecommons.org/licenses/by-sa/3.0/ ou envie uma carta para Creative Commons, PO Box 1866, Mountain View, CA 94042, USA.

%\documentclass[main.tex]{subfiles}
%\begin{document}

\chapter{Introdução}

Cálculo numérico é a disciplina que estuda as técnicas para a solução aproximada de problemas matemáticos. Estas técnicas são de natureza analítica e computacional. As principais preocupações normalmente envolvem exatidão e desempenho.

Aliado ao aumento contínuo da capacidade de computação disponível, o desenvolvimento de métodos numéricos tornou a simulação computacional\index{simulação!computacional} de problemas matemáticos uma prática usual nas mais diversas áreas científicas e tecnológicas. As então chamadas simulações numéricas\index{simulação!numérica} são constituídas de um arranjo de vários esquemas numéricos dedicados a resolver problemas específicos como, por exemplo: resolver equações algébricas, resolver sistemas de equações lineares, interpolar e ajustar pontos, calcular derivadas e integrais, resolver equações diferenciais ordinárias etc. Neste livro, abordamos o desenvolvimento, a implementação, a utilização e os aspectos teóricos de métodos numéricos para a resolução desses problemas.

Trabalharemos com problemas que abordam aspectos teóricos e de utilização dos métodos estudados, bem como com problemas de interesse na engenharia, na física e na matemática aplicada.

A necessidade de aplicar aproximações numéricas decorre do fato de que esses problemas podem se mostrar intratáveis se dispomos apenas de meios puramente analíticos, como aqueles estudados nos cursos de cálculo e álgebra linear. Por exemplo, o teorema de Abel-Ruffini nos garante que não existe uma fórmula algébrica, isto é, envolvendo apenas operações aritméticas e radicais, para calcular as raízes de uma equação polinomial de qualquer grau, mas apenas casos particulares:
\begin{itemize}
\item Simplesmente isolar a incógnita para encontrar a raiz de uma equação do primeiro grau;
\item Fórmula de Bhaskara para encontrar raízes de uma equação do segundo grau;
\item Fórmula de Cardano para encontrar raízes de uma equação do terceiro grau;
\item Existe expressão para equações de quarto grau;
\item Casos simplificados de equações de grau maior que 4 onde alguns coeficientes são nulos também podem ser resolvidos.
\end{itemize}
Equações não polinomiais podem ser ainda mais complicadas de resolver exatamente, por exemplo:
\begin{equation}
\cos(x)=x\qquad \text{ou}\qquad xe^x= 10
\end{equation}

% exemplos ---- -  -- - --  - - -














% exemplos ---- -  -- - --  - - -
A maioria dos problemas envolvendo fenômenos reais produzem modelos matemáticos cuja solução analítica é difícil (ou impossível) de obter, mesmo quando provamos que a solução existe. Nesse curso propomos calcular aproximações numéricas para esses problemas, que apesar de, em geral, serem diferentes da solução exata, mostraremos que elas podem ser bem próximas.

Para entender a construção de aproximações é necessário estudar como funciona a aritmética implementada nos computadores e erros de arredondamento. Como computadores, em geral, usam uma base binária para representar números, começaremos falando em mudança de base.


%\end{document}
